%COMANDI

% ===========================================================================================
% COMANDI FIRST PAGE
% ===========================================================================================

% Nome Versione del documento
\newcommand{\documentName}{}
\newcommand{\documentVersion}{}
\newcommand{\documentDate}{}

% Approvatore del documento
\newcommand{\documentApprovers}{}

% Editori del documento
\newcommand{\documentEditors}{}

% Verificatori del documento
\newcommand{\documentVerifiers}{}

% Uso del documento
\newcommand{\documentUsage}{}

% Destinatari del documento
\newcommand{\documentAddressee}{}

% Sommario del documento
\newcommand{\documentSummary}{}

% Versione dei documenti
\newcommand{\docVersionGlo}{\textit{}} % Glossario
\newcommand{\docVersionSdC}{\textit{1.0.0}} % Studio dei Capitolati
\newcommand{\docVersionPdP}{\textit{1.0.0}} % Piano di Progetto
\newcommand{\docVersionLC}{\textit{1.0.0}} % Lettera di Candidatura

% ===========================================================================================
% COMANDI CONTENUTO
% ===========================================================================================

% Stile font
\newcommand{\glo}{\ped{\textbf{\tiny G}}} % Testo glossario

% Nome del progetto
\newcommand{\projectName}{SWE}
\newcommand{\groupEmail}{\textit{pentasoftswe@gmail.com}}
\newcommand{\groupName}{\textit{PentaSoft}}

% Referenti e committente
\newcommand{\proposerName}{}
\newcommand{\commitNameM}{Prof. Tullio Vardanega}
\newcommand{\commitNameS}{Prof. Riccardo Cardin}

% Nome dei documenti
\newcommand{\docNameGlo}{\textit{Glossario}}
\newcommand{\docNameSdC}{\textit{Studio dei capitolati}}
\newcommand{\docNamePdP}{\textit{Piano di Progetto}}
\newcommand{\docNameLC}{\textit{Lettera di Candidatura}}

% Nome e versione dei documenti
\newcommand{\docNameVersionGlo}{\docNameGlo{} \docVersionGlo}

% Componenti del gruppo
\newcommand{\team}{Marco Rosin, Pietro Lauriola, Marco Brugin, Luca Marcato, Stefano Meneguzzo, Nicola Lazzarin}

% Descrizione del glossario
\newcommand{\gloDesc}{Alcuni dei termini utilizzati in questo documento potrebbero generare dei dubbi riguardo al loro significato, al fine di evitare tali ambiguità è necessario dar loro una definizione. Tali termini vengono contassegnati da una G maiuscola finale a pedice della parola. La loro spiegazione è riportata nel \docNameVersionGlo{}}

