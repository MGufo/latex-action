\section{Impegni e Scadenze}\label{section:impegni_scadenze}

\subsection{Prospetto orario e stima costi}\label{section:prospetto_orario}

A seguito dell'iniziale analisi del capitolato scelto, il gruppo ha stimato la seguente ripartizione
oraria di ogni ruolo:

\begin{center}
    \begin{tabular}{|c|c|c|c|}
        \hline
        Ruolo          & Ore(h) & Costo Orario(\euro{}\slash h) & Costo(\euro{}) \\
        Responsabile   & 45     & 30                            & 1350           \\
        Amministratore & 45     & 20                            & 900            \\
        Analista       & 80     & 25                            & 2000           \\
        Progettista    & 110    & 25                            & 2750           \\
        Programmatore  & 120    & 15                            & 1800           \\
        Verificatore   & 140    & 15                            & 2100           \\
        \hline
        Totale         & 540    &                               & 10900          \\
        \hline
    \end{tabular}
\end{center}

Dividendo il monte ore totale per il numero di membri del gruppo ne consegue che l'impegno di ogni membro si
può quantificare in \textbf{90 ore} di lavoro.

\subsection{Rotazione ruoli}\label{section:rotazione_ruoli}

Per garantire che ogni membro del gruppo ricopra almeno una volta ogni ruolo si è deciso di implementare
una rotazione dei ruoli di tipo \textit{round robin}, con \textit{clock} di rotazione da definire a
seguito dell'assegnazione del capitolato.

\subsection{Scadenze}\label{section:scadenze}

Il gruppo prevede di consegnare il prodotto finito entro il \textbf{01-05-2023} e stima di effettuare
le revisioni intermedie di avanzamento nelle seguenti date:

\begin{enumerate}
    \item \textbf{RTB} in data 10-01-2023.
    \item \textbf{PB} in data 15-03-2023.
    \item \textbf{CA} in data 20-04-2023.
\end{enumerate}

Al fine di monitorare con attenzione lo svolgimento delle attività di progetto il gruppo ha deciso di
imporsi \textit{milestones} a cadenza settimanale.