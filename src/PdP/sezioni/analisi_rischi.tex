\section{Analisi dei rischi}\label{section:analisi_rischi}

Durante lo sviluppo di un progetto vi è il rischio di incorrere in problemi, più o meno gravi, causati da un insieme di persone che possono rallentare e/o ostacolare il lavoro.
Per questo viene effettuata un’attenta analisi dei fattori di rischio. Per individuarli e gestirli al meglio vengono delineate le seguenti attività:

\begin{enumerate}
    \item \textbf{Identificazione}:  Vengono individuati i fattori di rischio che possono sorgere durante lo svolgimento del progetto.
    \item  \textbf{Analisi}: Vengono studiati i fattori di rischio individuati nell'attività precedente al fine di determinarne probabilità, gravità, potenziale impatto sul progetto e possibili soluzioni.
    \item \textbf{Pianificazione}: Viene stabilito un metodo che tenti di evitare i rischi identificati e definisca una strategia da applicare nel caso si verifichino.
    \item \textbf{Monitoraggio}: Vengono definite le azioni che i membri del gruppo devono rispettare per minimizzare la possibilità di incorrere nei rischi delineati e quali azioni svolgere nel caso in cui questi si realizzino.
\end{enumerate}

\subsection{Classificazione}\label{subsection:classificazione_rischi}

Al fine di facilitarne tracciamento e riferimento in altri documenti ogni rischio tracciato viene identificato
usando la seguente codifica:

\begin{itemize}
    \item \textit{Codice Identificativo}: Definito come \texttt{R[Tipologia][ID]}. Le tipologie attualmente tracciate sono:
          \begin{itemize}
              \item \texttt{T}: Tecnologico.
              \item \texttt{O}: Organizzativo.
              \item \texttt{I}: Interpersonale.
          \end{itemize}
    \item \textit{Titolo}
    \item \textit{Descrizione}
    \item \textit{Modalità di controllo}: Definisce la figura responsabile di controllare e monitorare il rischio e quale piano d'azione implementare per evitare che tale rischio si presenti.
    \item \textit{Piano di Contingenza}: Definisce cosa fare per mitigare il rischio nel caso in cui si presenti.
    \item \textit{Probabilità}
    \item \textit{Rilevanza}: Gravità del rischio, può essere:
          \begin{itemize}
              \item \textit{Accettabile}: Danno lieve, sostenibile senza risoluzione.
              \item \textit{Tollerabile}: Danno modesto, non richiede una risoluzione urgente.
              \item \textit{Inaccettabile}: Danno grave, necessita risoluzione il prima possibile.
          \end{itemize}
\end{itemize}