\section{Capitolato C6 - ShowRoom }\label{section:c6}

\subsection{Informazioni generali}
\begin{itemize}
    \item \textbf{Nome:} \textit{ShowRoom}
    \item \textbf{Proponente:} \textit{SanMarco Informatica}
    \item \textbf{Committente:} \commitNameM{} e \commitNameS{}
\end{itemize}

\subsection{Descrizione del capitolato}
L'obiettivo del capitolato è quello di realizzare uno showroom virtuale, accessibile tramite browser, per presentare prodotti ai clienti in un contesto altamente immersivo.
Tramite l'applicazione deve essere possibile effettuare, anche, le seguenti attività:
\begin{itemize}
    \item Muoversi all'interno dello showroom.
    \item Visualizzare i prodotti esposti.
    \item Visualizzare in dettaglio un prodotto, le sue caratteristiche ed eventuali varianti.
    \item Aggiungere un prodotto al carrello.
    \item Spostare un prodotto in aree diverse dello showroom.
\end{itemize}

\subsection{Finalità del progetto}
L’azienda vuole fornire agli utenti la possibilità di percepire le qualità dei prodotti esposti virtualmente,
di modificarne le caratteristiche in tempo reale e, se interessati, di aggiungerli ad un carrello per un
successivo acquisto.

\subsection{Tecnologie interessate}
Il proponente ha consigliato le seguenti tecnologie:
\begin{itemize}
    \item \textit{Three.js}, una libreria \textit{JavaScript} per lo sviuppo di applicazioni grafiche 3D.
    \item \textit{Unity 3D} e/o \textit{Unreal Engine}, motori grafici potenti e complessi, consigliati solo nel caso in cui i membri del gruppo avessero già dimestichezza con essi.
\end{itemize}

\subsection{Aspetti positivi}
\begin{itemize}
    \item Progetto innovativo, permette di trattare argomenti relativamente recenti
    \item Risulta molto interessante la possibilità di usare tecnologie di grafica 3D.
    \item La Proponente è molto disponibile per fornire supporto al gruppo.
\end{itemize}


\subsection{Criticità e fattori rischio}
Il capitolato proposto si focalizza principalmente sull'aspetto grafico, fuorviando eccessivamente dallo scopo del corso di Ingegneria del Software.

\subsection{Conclusione}
Per quanto il progetto sia inizialmente apparso interessante, è stato scartato in quanto ritenuto poco
affine all'area di competenza del corso.
