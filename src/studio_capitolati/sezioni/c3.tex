\section{Capitolato C3 - Personal Identity Wallet}\label{section:c3}

\subsection{Informazioni generali}
    \begin{itemize}
        \item \textbf{Nome:} \textit{Personal Identity Wallet}
        \item \textbf{Proponente:} \textit{InfoCert S.p.A}
        \item \textbf{Committente:} \commitNameM{} e \commitNameS{}
    \end{itemize}

\subsection{Descrizione del capitolato}
L’identità digitale è un cardine della vita di tutti i giorni: basti pensare a quanti siti web, applicazioni e servizi online consentano il login tramite account Google, Facebook, ecc; questi account tuttavia non possiedono valore giuridico tale da poter essere usati in contesti formali quali portali assicurativi, bancari, sanitari, ecc. A soluzione del problema ogni Stato ha introdotto un proprio servizio di identità digitale nazionale (es: SPID e CIE), non sempre compatibile con servizi analoghi usati da Stati esteri.\newline
Al fine di garantire interoperabilità tra credenziali nazionali e servizi digitali di Stati esteri l’UE ha recentemente proposto la creazione di una nuova "Identità Digitale", che garantisca l’autenticazione presso qualsiasi servizio digitale fornito all’interno dell’UE.

\subsection{Finalità del progetto}
Scopo del progetto è la realizzazione di un prototipo di sistema per la creazione, verifica e utilizzo di "credenziali verificabili".\newline
Il sistema deve quindi fornire agli utenti le seguenti caratteristiche:
    \begin{itemize}
        \item \textit{Generazione delle credenziali:}
            \begin{itemize}
                \item L’utente deve poter richiedere una credenziale ad un “ente certificatore” tramite apposita pagina sul sito web di quest’ultimo.
                \item L’impiegato  dell’ente certificatore deve poter autorizzare il rilascio della credenziale all’utente che ne ha fatto richiesta.
                \item L’utente deve poter ricevere la credenziale richiesta dal sito web dell’ente certificatore.
            \end{itemize}
        \item \textit{Gestione delle credenziali:}
            \begin{itemize}
                \item L’utente deve poter visualizzare, tramite apposita webapp,  le credenziali ricevute dai vari enti certificatori.
                \item L’utente deve poter eliminare, tramite apposita webapp, le credenziali ricevute dai vari enti certificatori.
            \end{itemize}
        \item \textit{Verifica delle credenziali:}
            \begin{itemize}
                \item L’utente deve poter esibire una credenziale contenuta nel suo wallet ad un verificatore che ne faccia richiesta.
                \item Un verificatore deve poter richiedere l’esibizione di una credenziale ad un utente che stia navigando sul sito web del verificatore.
                \item Un verificatore deve poter certificare la correttezza della credenziale esibita dall'utente.
            \end{itemize}
    \end{itemize}

\subsection{Tecnologie interessate}
La proponente non ha delineato uno stack tecnologico nella presentazione del capitolato, lasciando l’onere della scelta motivata al fornitore.

\subsection{Aspetti positivi}
    \begin{itemize}
        \item Il progetto verte su temi attuali e di pubblico interesse.
        \item Il prototipo realizzato potrebbe essere esteso o usato come PoC da cui partire per realizzare quanto richiesto dall'UE, aggiungendo valore al C.V. dei membri del gruppo fornitore.
    \end{itemize}

\subsection{Criticità e fattori rischio}
Alla data di scrittura di questa analisi la specifica tecnica riguardante il formato delle credenziali e i protocolli di comunicazione non è stata resa disponibile dalla UE. Ciò potrebbe causare difficoltà e/o ritardi aggiuntivi nel tracciamento dei requisiti e nella realizzazione del progetto.

\subsection{Conclusione}
Pur apprezzandone la nobile finalità il team di sviluppo ha dimostrato scarso interesse nella realizzazione del capitolato, preferendogli proposte più stimolanti dal punto di vista tecnologico.

