\section{Capitolato C7 - Trustify Authentic and verifiable reviews platform}\label{section:c7}

\subsection{Informazioni generali}
\begin{itemize}
    \item \textbf{Nome:} \textit{Trustify - Authentic and verifiable reviews platform}
    \item \textbf{Proponente:} \textit{Synclab}
    \item \textbf{Committente:} \commitNameM{} e \commitNameS{}
\end{itemize}

\subsection{Descrizione del capitolato}
Nel contesto delle recensioni online esiste attualmente un problema di autenticità: le recensioni presenti
sul sito di un’attività non sono effettivamente verificabili.
Esiste infatti un fenomeno chiamato \textit{review bombing}, per il quale le recensioni possono essere
facilmente falsificate e rilasciate in massa pur non essendo legate ad un acquisto realmente avvenuto.

\subsection{Finalità del progetto}
L'obiettivo del capitolato è la realizzazione di una \textit{webapp} per fornire un servizio di pagamento
e recensione basata sull'utilizzo di \textit{smart contract}, per loro natura immutabili e pubblicamente verificabili.\newline
La \textit{webapp} dovrà inoltre consentire la visualizzazione delle recensioni rilasciate, organizzate per attività commerciale.

\subsection{Tecnologie interessate}
L’azienda propone alcune scelte preferenziali da considerare, in particolare:
\begin{itemize}
    \item Scelta di una \textit{Blockchain Ethereum}-compatibile.
    \item Linguaggio di programmazione \textit{Solidity} per la scrittura dello \textit{smart contract}.
    \item \textit{Framework Java Spring} per lo sviluppo del servizio \textit{API REST}.
    \item \textit{Framework Angular JS} per lo sviluppo della \textit{webapp}.
    \item Librerie \textit{web3js} e \textit{web3j} per le interazioni con lo \textit{smart contract}.
    \item Utilizzo di un fornitore terzo per \textit{RPC} a nodo.
    \item Utilizzo del \textit{wallet Metamask} per la firma delle transazioni da parte degli utenti.
\end{itemize}

\subsection{Aspetti positivi}
Il capitolato offre l’opportunità di conoscere l'ambiente innovativo delle \textit{blockchain} e di diffonderne l'uso al pubblico generale.
Offre inoltre la possibilità di apprendere nuove tecnologie che potranno essere utili nel mondo del lavoro e per una crescita professionale dei membri del gruppo.\newline
I proponenti si sono dimostrati disponibili a seguirci durante lo svolgimento del capitolato, mettendo a disposizione figure professionali con diversi anni di esperienza.


\subsection{Criticità e fattori rischio}
Nonostante l’utilizzo di tecnologie mai usate rappresenti un’opportunità come prima citato, rappresenta allo stesso tempo una grande sfida per l’intero gruppo.
\subsection{Conclusione}
In conclusione il gruppo ritiene gli che aspetti positivi sopra elencati abbiano un peso molto più consistente rispetto alle criticità.
Inoltre in seguito al confronto con l’azienda il gruppo ha manifestato un vivo interesse riguardo al tema del progetto e ha trovato grande disponibilità da parte dei proponenti.
