\section{Capitolato C1 - CAPTCHA: Umano o Sovrumano}\label{section:c1}

\subsection{Informazioni generali}
\begin{itemize}
    \item \textbf{Nome:} \textit{CAPTCHA: Umano o Sovrumano}
    \item \textbf{Proponente:} \textit{Zucchetti S.p.A}
    \item \textbf{Committente:} \commitNameM{} e \commitNameS{}
\end{itemize}

\subsection{Descrizione del capitolato}
Oggigiorno, considerando il crescente impiego di mezzi informatici e sistemi cloud nella vita di tutti i giorni diventa fondamentale l’impiego di strumenti (\textit{Captcha}) che permettano la distinzione tra persone fisiche e comportamenti robotizzati di strumenti automatici.
All'avanzare del progresso tecnologico sono stati creati \textit{Captcha} sempre più avanzati nel
tentativo di contrastare \textit{bot} in grado di risolvere i \textit{Captcha} precedentemente in uso.


\subsection{Finalità del progetto}
Obiettivo di questo capitolato è la realizzazione di una pagina web contenente uno strumento
\textit{Captcha} in grado di valutare se il tentativo di accesso sia stato svolto da un essere umano o un robot.
Viene anche richiesto di analizzare la resistenza del sistema \textit{Captcha} a tentativi di accesso
automatico da parte di \textit{bot} esistenti e futuri.
Infine, come requisito opzionale l’azienda richiede pure la realizzazione di:

\begin{itemize}
    \item Un form di registrazione di un nuovo utente.
    \item Un mini-forum contenente contenuti prodotti dagli utenti registrati.
    \item Una funzione di ricerca sul forum con verifica tramite \textit{Captcha}.
\end{itemize}

\subsection{Tecnologie interessate}
L’azienda non ha posto vincoli architetturali, ma richiede l’utilizzo di \textit{HTML, CSS e JS}
per la parte \textit{client} e di \textit{Java} o \textit{PHP} per la parte \textit{server}.

\subsection{Aspetti positivi}
\begin{itemize}
    \item Progetto proiettato verso il futuro.
    \item Offre la possibilità di analizzare le vulnerabilità di un sistema di protezione.
    \item Analizzare e comprendere le differenti interazioni con le immagini da parte di esseri umani e software.
\end{itemize}

\subsection{Criticità e fattori di rischio}
\begin{itemize}
    \item Difficoltà nel garantire l'accessibilità del sistema a portatori di disabilità non ancora identificate.
    \item Difficoltà nello stimare con precisione la resistenza del prodotto a tecnologie di attacco automatico non ancora esistenti.
\end{itemize}

\subsection{Conclusione}
Considerato il progetto nella sua interezza e valutata la difficoltà non banale dell'analisi da produrre,
la maggioranza del gruppo ha deciso di scartare il capitolato in quanto considerato
non sufficientemente interessante.
