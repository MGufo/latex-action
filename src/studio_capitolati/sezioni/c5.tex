\section{Capitolato C5 - SmartLog}\label{section:c5}

\subsection{Informazioni generali}
\begin{itemize}
    \item \textbf{Nome:} \textit{SmartLog}
    \item \textbf{Proponente:} \textit{Socomec}
    \item \textbf{Committente:} \commitNameM{} e \commitNameS{}
\end{itemize}

\subsection{Descrizione del capitolato}
L’obiettivo di questo capitolato è quello di realizzare due applicazioni, tramite delle “viste” e una serie
di strumenti statistici, per facilitare e velocizzare l'analisi di \textit{file log} ai tecnici della manutenzione.\newline
Le due applicazioni dovranno avere queste caratteristiche:
\begin{itemize}
    \item Interfaccia di visualizzazione di tipo Web.
    \item Caricamento e visualizzazione di un singolo file di log (.csv).
    \item Una visualizzazione in forma tabellare con relative funzioni di filtro, ricerca e ordinamento
    \item Una visualizzazione in forma grafica con relative funzioni di select, zoom, ecc.
    \item Una funzione di ricerca sequenze eventi note all’interno di un log
    \item Una selezione dei log da analizzare per range di data/ora
    \item Visualizzazione di varie statistiche in formato tabellare
\end{itemize}

\subsection{Finalità del progetto}
L’azienda richiede che ogni volta un tecnico interviene su una loro apparecchiatura, le informazioni (LOG), possono essere scaricate sul PC in formato .csv per poi poter essere analizzate nella webapp in formato “vista”.

\subsection{Tecnologie interessate}
Il committente non ha imposto vincoli stretti riguardo le tecnologie da usare, ma ha suggerito:
\begin{itemize}
    \item \textit{Python} per analisi statistica e gestione dei dati estratti dai \textit{log}.
    \item La libreria \textit{JavaScript D3.js} per creare visualizzazioni dinamiche ed interattive dei dati analizzati.
\end{itemize}

\subsection{Aspetti positivi}
Il progetto potrebbe essere utile all’azienda proponente.

\subsection{Criticità e fattori rischio}
Il capitolato si focalizza maggiormente sull'analisi e visualizzazione dei dati più che sullo sviluppo dell’applicativo web.

\subsection{Conclusione}
Nonostante l’iniziale interesse del gruppo il capitolato non è stato scelto perché a seguito di un'analisi
approfondita non è risultato sufficientemente interessante ai membri del gruppo.

