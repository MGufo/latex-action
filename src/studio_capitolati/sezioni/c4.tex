\section{Capitolato C4 - Piattaforma di localizzazione testi}\label{section:c4}

\subsection{Informazioni generali}
\begin{itemize}
    \item \textbf{Nome:} \textit{Piattaforma di localizzazione testi}
    \item \textbf{Proponente:} \textit{Zero 12}
    \item \textbf{Committente:} \commitNameM{} e \commitNameS{}
\end{itemize}

\subsection{Descrizione del capitolato}
Con la diffusione del web in differenti paesi del mondo è sorto il problema di rendere disponibile un
certo contenuto ad un pubblico sempre più vasto e variegato.
Risulta necessario in tal senso andare ad abbattere le barriere linguistiche per rendere l’applicazione
di nostro interesse il più possibile competitiva in un mercato internazionale enorme come il nostro, ed andare quindi ad attingere ad un bacino di utenza, e quindi di possibili clienti, sensibilmente maggiore.\newline
Nasce così l’idea di questa piattaforma che si pone proprio l'obiettivo di rendere disponibili traduzioni di determinati contenuti in diverse lingue, proprio con lo scopo di permettere a colui che ne voglia sfruttare le funzionalità di ritagliarsi una ferma il più grande possibile di mercato.

\subsection{Finalità del progetto}
Lo scopo di qusto progetto è creare una piattaforma che permetta di gestire i testi delle localizzazioni di \textit{mobile app} e \textit{webapp}.
La piattaforma deve supportare il \textit{multi-tenant}, ovvero la capacità di fornire servizi per diverse organizzazioni mantenendo i corrispettivi dati isolati a livello logico tra di loro.\newline
Si richiede di considerare 2 tipologie di utenti:
\begin{itemize}
    \item \textit{admin user}: Gestore delle organizzazioni abilitate ad accedere alla piattaforma, rappresenta La Proponente.
    \item \textit{content user}: Utenti che si occuperanno di creare contenuti e traduzioni.
\end{itemize}

È infine richiesto che le traduzioni prevedano processi di approvazione, un concetto di versionamento, una gestione tramite chiave univoca e la possibilità di suddividerle in gruppi.

\subsection{Tecnologie interessate}
Per la realizzazione di questo progetto viene richiesto l’utilizzo di tecnologie di \textit{Amazon Web Services}, in particolar modo i servizi:
\begin{itemize}
    \item \textit{AWS fargate}: permette la gestione a \textit{container} senza l’utilizzo di server.
    \item \textit{AWS Aurora Serverless}: servizio serverless per la gestione di database \textit{SQL}.
\end{itemize}
Per quanto riguarda i linguaggi di programmazione si richiede l’uso di:
\begin{itemize}
    \item \textit{NodeJS} per sviluppare l’API Restful JSON di supporto.
    \item \textit{Typescript} per lo sviluppo di una libreria frontend.
    \item \textit{Swift} per lo sviluppo di una libreria iOS/MacOS.
    \item \textit{Kotlin} per lo sviluppo di una libreria Android.
\end{itemize}
\subsection{Aspetti positivi}
\begin{itemize}
    \item Il progetto offre l'opportunità di conoscere le tecnologie \textit{Amazon Web Services}, offrendo anche \$500 di crediti spendibili sulla piattaforma.
    \item L’azienda si è dimostrata molto disponibile, fornendo un corso di formazione per lo sviluppo \textit{cloud native} e di \textit{API Restful JSON}.
\end{itemize}

\subsection{Criticità e fattori rischio}
Il gruppo non ha manifestato particolare interesse nell’idea alla base di questo progetto, preferendo ad esso altri capitolati ritenuti più stimolanti.

\subsection{Conclusione}
Nonostante il progetto sia più che valido il gruppo ha preferito inserire in cima alla lista delle preferenze altri capitolati a seguito di un maggiore interesse riscontrato dai suoi componenti.

